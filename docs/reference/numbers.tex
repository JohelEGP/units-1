%!TEX root = std.tex
\Sec0[nums]{Numbers library}

\Sec1[nums.summary]{Summary}

\pnum
This Clause describes components for dealing with numbers,
as summarized in \tref{nums.summary}.

\begin{libsumtab}{Numbers library summary}{nums.summary}
\ref{num.traits}        & Traits                                 & \tcode{<mp-units/numbers.h>} \\
\ref{num.concepts}      & Concepts                               & \\
\end{libsumtab}

\Sec1[nums.syn]{Header \tcode{<mp-units/numbers.h>} synopsis}

\indexheader{mp-units/numbers.h}%
\begin{codeblock}
namespace mp_units {

// \ref{num.traits}, traits

// \ref{num.opt.ins}, \cname{number} opt-ins
template<typename T>
struct is_number;
template<typename T>
struct is_complex_number;

template<typename T>
constexpr bool @\libglobal{is_number_v}@ = is_number<T>::value;
template<typename T>
constexpr bool @\libglobal{is_complex_number_v}@ = is_complex_number<T>::value;

// \ref{num.ids}, identities
template<typename T>
struct number_zero;
template<typename T>
struct number_one;

template<typename T>
constexpr T @\libglobal{number_zero_v}@ = number_zero<T>::value;
template<typename T>
constexpr T @\libglobal{number_one_v}@ = number_one<T>::value;

// \ref{num.assoc.types}, associated types
template<typename T>
struct vector_scalar;

template<typename T>
using number_difference_t = decltype(std::declval<const T>() - std::declval<const T>());
template<typename T>
using @\libglobal{vector_scalar_t}@ = vector_scalar<T>::type;

// \ref{num.concepts}, concepts
template<typename T>
concept number = @\seebelow@;
template<typename T, typename U>
concept common_number_with = @\seebelow@;
template<typename T>
concept ordered_number = @\seebelow@;
template<typename T>
concept number_line = @\seebelow@;
template<typename T, typename U>
concept modulo_with = @\seebelow@;
template<typename T, typename U>
concept compound_modulo_with = @\seebelow@;
template<typename T, typename U>
concept modulus_for = @\seebelow@;
template<typename T, typename U>
concept compound_modulus_for = @\seebelow@;
template<typename T>
concept negative = @\seebelow@;
template<typename T>
concept set_with_inverse = @\seebelow@;
template<typename T, typename U>
concept point_space_for = @\seebelow@;
template<typename T, typename U>
concept compound_point_space_for = @\seebelow@;
template<typename T>
concept point_space = @\seebelow@;
template<typename T, typename U>
concept vector_space_for = @\seebelow@;
template<typename T, typename U>
concept compound_vector_space_for = @\seebelow@;
template<typename T, typename U>
concept scalar_for = @\seebelow@;
template<typename T, typename U>
concept field_for = @\seebelow@;
template<typename T, typename U>
concept compound_scalar_for = @\seebelow@;
template<typename T, typename U>
concept compound_field_for = @\seebelow@;
template<typename T>
concept vector_space = @\seebelow@;
template<typename T>
concept f_vector_space = @\seebelow@;
template<typename T>
concept field_number = @\seebelow@;
template<typename T>
concept field_number_line = @\seebelow@;
template<typename T>
concept scalar_number = @\seebelow@;

}  // namespace mp_units
\end{codeblock}

\Sec1[num.traits]{Traits}

\Sec2[num.traits.reqs]{Requirements}

\pnum
Subclause \ref{num.traits} subsumes \refcpp{meta.rqmts},
assumingly amended for its context.
Pursuant to the subsumed \refcpp{namespace.std}\iref{spec.ext},
each class template specified in \ref{num.traits}
may be specialized for any numeric type\irefcppx{numeric.requirements}{def:type,numeric}.

\pnum
\indexlibrary{\idxexposid{number-trait}!{\tcode{const T}}}%
Each template \tcode{\placeholder{number-trait}} specified in \ref{num.traits}
has a partial specialization of the form
\begin{codeblock}
template<class T>
struct @\placeholder{number-trait}@<const T> : @\placeholder{number-trait}@<T> { };
\end{codeblock}

\pnum
Each template \tcode{\placeholder{number-trait}} specified in \ref{num.opt.ins}
is a \oldconcept{UnaryTypeTrait}
with a base characteristic of \tcode{std::bool_constant<\placeholder{B}>}.
\tcode{\placeholder{B}} is a value consistent with \placeholdernc{\tcode{number-trait}}'s specification.

\pnum
Each template specified in \ref{num.ids}
is a numeric distinguished value trait
(\href{https://wg21.link/P1841R3}{P1841R3} [num.traits.val]),
except that there are no declared specializations for arithmetic or volatile-qualified types.

\pnum
A specialization of any template \tcode{\placeholder{number-trait}} specified in \ref{num.assoc.types}
has no members other than a \tcode{type} member
that names a type consistent with \placeholdernc{\tcode{number-trait}}'s specification.

\Sec2[num.opt.ins]{\cname{number} opt-ins}

\begin{itemdecl}
template<typename T>
struct @\libglobal{is_number}@ : @\seebelow@ {};
\end{itemdecl}

\begin{itemdescr}
\pnum
\cvalue
\tcode{true} if \tcode{T} represents a number, and
\tcode{false} otherwise.

\pnum
\dvalue
\tcode{true} if \tcode{vector_scalar_t<T>} is valid, and
\tcode{false} otherwise.
\end{itemdescr}

\indexlibraryspec{is_number}{std::chrono::time_point}%
\indexlibraryspec{is_number}{std::chrono::day}%
\indexlibraryspec{is_number}{std::chrono::month}%
\indexlibraryspec{is_number}{std::chrono::year}%
\indexlibraryspec{is_number}{std::chrono::weekday}%
\indexlibraryspec{is_number}{std::chrono::year_month}%
\begin{itemdecl}
template<class... T>
struct is_number<std::chrono::time_point<T...>> : std::true_type {};
template<>
struct is_number<std::chrono::day> : std::true_type {};
template<>
struct is_number<std::chrono::month> : std::true_type {};
template<>
struct is_number<std::chrono::year> : std::true_type {};
template<>
struct is_number<std::chrono::weekday> : std::true_type {};
template<>
struct is_number<std::chrono::year_month> : std::true_type {};
\end{itemdecl}

\begin{itemdecl}
template<typename T>
struct @\libglobal{is_complex_number}@ : std::false_type {};
\end{itemdecl}

\begin{itemdescr}
\pnum
\cvalue
\tcode{true} if \tcode{T} represents a complex number\irefiev{102-02-09}, and
\tcode{false} otherwise.
\end{itemdescr}

\indexlibraryspec{is_complex_number}{std::complex}%
\begin{itemdecl}
template<class T>
struct is_complex_number<std::complex<T>> : std::true_type {};
\end{itemdecl}

\Sec2[num.ids]{Identities}

\begin{codeblock}
template<typename T>
concept @\defexposconcept{inferable-identities}@ =
  common_number_with<T, number_difference_t<T>> && std::constructible_from<T, int>;
\end{codeblock}

\indexlibrary{\idxcode{number_zero}!\idxexposconcept{inferable-identities}}%
\begin{itemdecl}
template<typename T>
struct @\libglobal{number_zero}@ { @\seebelow@ };
\end{itemdecl}

\begin{itemdescr}
\pnum
\cvalue
\tcode{T}'s neutral element for addition\irefiev{102-01-12}, if any.

\pnum
\dvalue
If \tcode{T} models \exposconcept{inferable-identities}, \tcode{T(0)}.
\end{itemdescr}

\indexlibrary{\idxcode{number_one}!\idxexposconcept{inferable-identities}}%
\begin{itemdecl}
template<typename T>
struct @\libglobal{number_one}@ { @\seebelow@ };
\end{itemdecl}

\begin{itemdescr}
\pnum
\cvalue
\tcode{T}'s neutral element for multiplication\irefiev{102-01-19}, if any.

\pnum
\dvalue
If \tcode{T} models \exposconcept{inferable-identities}, \tcode{T(1)}.
\end{itemdescr}

\Sec2[num.assoc.types]{Associated types}

\begin{itemdecl}
template<typename T>
using @\libglobal{number_difference_t}@ = decltype(std::declval<const T>() - std::declval<const T>());
\end{itemdecl}

\begin{itemdescr}
\pnum
\tcode{number_difference_t<T>} represents \tcode{T}'s difference's\irefiev{102-01-17} type, if any.
\end{itemdescr}

\indexlibrary{\idxcode{vector_scalar}!arithmetic type}%
\indexlibraryspec{vector_scalar}{std::complex}%
\indexlibraryspec{vector_scalar}{std::chrono::duration}%
\begin{itemdecl}
template<typename T>
struct @\libglobal{vector_scalar}@ {};
template<@\seebelow@>
struct vector_scalar<@\seebelow@> {
  using type = @\seebelow@;
};
\end{itemdecl}

\begin{itemdescr}
\pnum
\ctype
Scalar\irefiev{102-02-18} for the vector space\irefiev{102-03-01} \tcode{T}.

\pnum
Each row of \tref{num.scalar} denotes a specialization.

\begin{floattable}{Specializations of \tcode{vector_scalar}}{num.scalar}{lll}
\topline
\lhdr{\fakegrammarterm{template-parameter-list}}
                              & \chdr{\fakegrammarterm{template-argument-list}}
                                                                    & \rhdr{\Fundescx{Type}} \\ \rowsep
\tcode{std::integral T}       & \tcode{T}                           & \tcode{T} \\ \rowsep
\tcode{std::floating_point T} & \tcode{T}                           & \tcode{T} \\ \rowsep
\tcode{class T}               & \tcode{std::complex<T>}             & \tcode{T} \\ \rowsep
\tcode{class T, class U}      & \tcode{std::chrono::duration<T, U>} & \tcode{T} \\
\end{floattable}
\end{itemdescr}

\Sec1[num.concepts]{Concepts}

\begin{itemdecl}
template<typename T>
concept @\deflibconcept{number}@ = is_number_v<T> && std::regular<T>;

template<typename T, typename U>
concept @\deflibconcept{common_number_with}@ =
  @\libconcept{number}@<T> && @\libconcept{number}@<U> && std::common_with<T, U> && @\libconcept{number}@<std::common_type_t<T, U>>;

template<typename T>
concept @\deflibconcept{ordered_number}@ = @\libconcept{number}@<T> && std::totally_ordered<T>;
\end{itemdecl}

\begin{itemdecl}
template<class T>
concept @\deflibconcept{number_line}@ =
  @\libconcept{ordered_number}@<T> &&
  requires(T& v) {
    number_one_v<number_difference_t<T>>;
    { ++v } -> std::same_as<T&>;
    { --v } -> std::same_as<T&>;
    { v++ } -> std::same_as<T>;
    { v-- } -> std::same_as<T>;
  };
\end{itemdecl}

\begin{itemdescr}
\pnum
Let \tcode{v} and \tcode{u} be equal objects of type \tcode{T},
and \tcode{one} be the value \tcode{number_one_v<number_difference_t<T>>}.
\tcode{T} models \libconcept{number_line} only if
\begin{itemize}
\item
The expressions \tcode{++v} and \tcode{v++} have the same domain\irefcppx{concepts.equality}{def:expression,domain}.
\item
The expressions \tcode{--v} and \tcode{v--} have the same domain.
\item
If \tcode{v} is incrementable\irefcppx{iterator.concept.winc}{\#def:incrementable},
then both \tcode{++v} and \tcode{v++} add \tcode{one} to \tcode{v},
and the following expressions all equal \tcode{true}:
\begin{itemize}
\item \tcode{v++ == u},
\item \tcode{((void)v++, v) == ++u}, and
\item \tcode{std::addressof(++v) == std::addressof(v)}.
\end{itemize}
\item
If \tcode{v} is decrementable\irefcppx{range.iota.view}{\#def:decrementable},
then both \tcode{--v} and \tcode{v--} subtract \tcode{one} from \tcode{v},
and the following expressions all equal \tcode{true}:
\begin{itemize}
\item \tcode{v-- == u},
\item \tcode{((void)v--, v) == --u}, and
\item \tcode{std::addressof(--v) == std::addressof(v)}.
\end{itemize}
\end{itemize}
\end{itemdescr}

\begin{itemdecl}
template<class T, class U> concept @\defexposconceptnc{addition-with}@ =
  @\libconcept{number}@<T> &&
  @\libconcept{number}@<U> &&
  requires(const T& c, const U& d) {
    { c + d } -> common_number_with<T>;
    { d + c } -> common_number_with<T>;
  };

template<class T, class U> concept @\defexposconceptnc{compound-addition-with}@ =
  @\exposconcept{addition-with}@<T, U> &&
  requires(T& l, const U& d) {
    { l += d } -> std::same_as<T&>;
  };

template<class T, class U> concept @\defexposconceptnc{subtraction-with}@ =
  @\exposconcept{addition-with}@<T, U> &&
  requires(const T& c, const U& d) {
    { c - d } -> common_number_with<T>;
  };

template<class T, class U> concept @\defexposconceptnc{compound-subtraction-with}@ =
  @\exposconcept{subtraction-with}@<T, U> &&
  @\exposconcept{compound-addition-with}@<T, U> &&
  requires(T& l, const U& d) {
    { l -= d } -> std::same_as<T&>;
  };

template<class T, class U, class V> concept @\defexposconceptnc{multiplication-with}@ =
  @\libconcept{number}@<T> &&
  @\libconcept{number}@<U> &&
  @\libconcept{number}@<V> &&
  requires(const T& c, const U& u) {
    { c * u } -> common_number_with<V>;
  };

template<class T, class U> concept @\defexposconceptnc{compound-multiplication-with}@ =
  @\exposconcept{multiplication-with}@<T, U, T> &&
  requires(T& l, const U& u) {
    { l *= u } -> std::same_as<T&>;
  };

template<class T, class U> concept @\defexposconceptnc{division-with}@ =
  @\exposconcept{multiplication-with}@<T, U, T> &&
  @\exposconcept{multiplication-with}@<U, T, T> &&
  requires(const T& c, const U& u) {
    { c / u } -> common_number_with<T>;
  };

template<class T, class U> concept @\defexposconceptnc{compound-division-with}@ =
  @\exposconcept{division-with}@<T, U> &&
  @\exposconcept{compound-multiplication-with}@<T, U> &&
  requires(T& l, const U& u) {
    { l /= u } -> std::same_as<T&>;
  };

template<class T, class U> concept @\deflibconcept{modulo_with}@ =
  @\libconcept{number}@<T> &&
  @\libconcept{number}@<U> &&
  requires(const T& c, const U& u) {
    { c % u } -> common_number_with<T>;
  };

template<class T, class U> concept @\deflibconcept{compound_modulo_with}@ =
  @\libconcept{modulo_with}@<T, U> &&
  requires(T& l, const U& u) {
    { l %= u } -> std::same_as<T&>;
  };
\end{itemdecl}

\begin{itemdescr}
\pnum
Let \tcode{q} be an object of type \tcode{T},
and \tcode{r} be an object of type \tcode{U}.

\pnum
For \exposconcept{addition-with} and \exposconcept{compound-addition-with},
let $E$ be the expression \tcode{q + r} or \tcode{r + q}, and \tcode{q += r}, respectively, and
let $F$ be the addition\irefiev{102-01-11} of the inputs to $E$.

\pnum
For \exposconcept{subtraction-with} and \exposconcept{compound-subtraction-with},
let $E$ be the expression \tcode{q - r} and \tcode{q -= r}, respectively, and
let $F$ be the subtraction\irefiev{102-01-13} of the inputs to $E$.

\pnum
For \exposconcept{multiplication-with} and \exposconcept{compound-multiplication-with},
let $E$ be the expression \tcode{q * r} and \tcode{q *= r}, respectively, and
let $F$ be the multiplication\irefiev{102-01-18} of the inputs to $E$.

\pnum
For \exposconcept{division-with} and \exposconcept{compound-division-with},
let $E$ be the expression \tcode{q / r} and \tcode{q /= r}, respectively, and
let $F$ be the division\irefiev{102-01-21} of the inputs to $E$.

\pnum
For \libconcept{modulo_with} and \libconcept{compound_modulo_with},
let $E$ be the expression \tcode{q \% r} and \tcode{q \%= r}, respectively, and
let $F$ be the modulo\iref{def.mod} of the inputs to $E$.

\pnum
\tcode{T} respectively models
\tcode{\exposconceptnc{addition-with}<U>},
\tcode{\exposconceptnc{compound-addition-with}<U>},
\tcode{\exposconceptnc{subtraction-with}<U>},
\tcode{\exposconceptx{com\-pound-subtraction-with}{compound-subtraction-with}<U>},
\tcode{\exposconceptnc{multiplication-with}<U, V>},
\tcode{\exposconceptnc{compound-multiplication-with}<U>},
\tcode{\exposconceptnc{division-with}<U>},
\tcode{\exposconceptnc{compound-division-with}<U>},
\tcode{\libconcept{modulo_with}<U>}, and
\tcode{\libconcept{compound_modulo_with}<U>}
only if, for each respective $E$, when the inputs to $E$ are in the domain of $E$
\begin{itemize}
\item
$E$ performs $F$ in an unspecified set.
\item
If the operator of $E$ is an \fakegrammarterm{assignment-operator},
\begin{itemize}
\item
$E$ maps the value of $F$ to \tcode{q}, and
the result of $E$ is a reference to \tcode{q}, and
\item
the result of $E$ is the value of $F$ mapped to the type of $E$ otherwise.
\end{itemize}
\end{itemize}
\end{itemdescr}

\begin{itemdecl}
template<typename T, typename U>
concept @\deflibconcept{modulus_for}@ = @\libconcept{modulo_with}@<U, T>;
template<typename T, typename U>
concept @\deflibconcept{compound_modulus_for}@ = @\libconcept{compound_modulo_with}@<U, T>;
\end{itemdecl}

\begin{itemdecl}
template<class T> concept @\deflibconcept{negative}@ =
  @\exposconcept{compound-addition-with}@<T, T> &&
  requires(const T& c) {
    number_zero_v<T>;
    { -c } -> common_number_with<T>;
  };

template<class T> concept @\deflibconcept{set_with_inverse}@ =
  @\exposconcept{compound-multiplication-with}@<T, T> &&
  requires(const T& c) {
    { number_one_v<T> / c } -> std::common_with<T>;
  };
\end{itemdecl}

\begin{itemdescr}
\pnum
Let \tcode{v} be an object of type \tcode{T}.

\pnum
For \libconcept{negative},
let $E$ be the expression \tcode{-v}, and
let $F$ be the negative\irefiev{102-01-14} of \tcode{v}.

\pnum
For \libconcept{set_with_inverse},
let $E$ be the expression \tcode{number_one_v<T> / v}, and
let $F$ be the inverse\irefiev{102-01-24} of \tcode{v}.

\pnum
\tcode{T} respectively models
\tcode{\libconcept{negative}<U>} and
\tcode{\libconcept{set_with_inverse}<U>}
only if, for the respective $E$, when \tcode{v} is in the domain of $E$
\begin{itemize}
\item
$E$ performs $F$ in an unspecified set.
\item
The result of $E$ is the value of $F$ mapped to the type of $E$.
\end{itemize}
\end{itemdescr}

\begin{itemdecl}
template<typename T, typename U>
concept @\deflibconcept{point_space_for}@ =
  @\exposconcept{subtraction-with}@<T, U> && @\libconcept{negative}@<U> && common_number_with<number_difference_t<T>, U>;
template<typename T, typename U>
concept @\deflibconcept{compound_point_space_for}@ = @\libconcept{point_space_for}@<T, U> && @\exposconcept{compound-subtraction-with}@<T, U>;
template<typename T>
concept @\deflibconcept{point_space}@ = @\libconcept{compound_point_space_for}@<T, number_difference_t<T>>;
\end{itemdecl}

\begin{itemdescr}
\pnum
\begin{note}
The \libconcept{point_space} concept is modeled by types
that behave similarly to \tcode{std::chrono::sys_seconds}.
\end{note}
\end{itemdescr}

\begin{itemdecl}
template<typename T, typename U>
concept @\deflibconcept{vector_space_for}@ = @\libconcept{point_space_for}@<U, T>;
template<typename T, typename U>
concept @\deflibconcept{compound_vector_space_for}@ = @\libconcept{compound_point_space_for}@<U, T>;

template<typename T>
concept @\defexposconceptnc{weak-scalar}@ =
  common_number_with<T, number_difference_t<T>> && @\libconcept{point_space}@<T> && @\libconcept{negative}@<T>;

template<typename T, typename U>
concept @\defexposconceptnc{scales-with}@ = common_number_with<U, vector_scalar_t<T>> && @\exposconcept{weak-scalar}@<U> &&
                      @\exposconcept{multiplication-with}@<T, U, T> && @\libconcept{set_with_inverse}@<U>;
template<typename T, typename U>
concept @\defexposconceptnc{compound-scales-with}@ = @\exposconcept{scales-with}@<T, U> && @\exposconcept{compound-multiplication-with}@<T, U>;

template<typename T, typename U>
concept @\deflibconcept{scalar_for}@ = @\exposconcept{scales-with}@<U, T>;
template<typename T, typename U>
concept @\deflibconcept{field_for}@ = @\libconcept{scalar_for}@<T, U> && @\exposconcept{division-with}@<U, T>;
template<typename T, typename U>
concept @\deflibconcept{compound_scalar_for}@ = @\exposconcept{compound-scales-with}@<U, T>;
template<typename T, typename U>
concept @\deflibconcept{compound_field_for}@ = @\libconcept{compound_scalar_for}@<T, U> && @\exposconcept{compound-division-with}@<U, T>;

template<typename T>
concept @\deflibconcept{vector_space}@ = @\libconcept{point_space}@<T> && @\exposconcept{compound-scales-with}@<T, vector_scalar_t<T>>;
template<typename T>
concept @\deflibconcept{f_vector_space}@ = @\libconcept{vector_space}@<T> && @\exposconcept{compound-division-with}@<T, vector_scalar_t<T>>;
\end{itemdecl}

\begin{itemdescr}
\pnum
\begin{note}
The \libconcept{f_vector_space} concept is modeled by types
that behave similarly to \tcode{std::chrono::seconds}.
\end{note}
\end{itemdescr}

\begin{itemdecl}
template<typename T>
concept @\deflibconcept{field_number}@ = @\libconcept{f_vector_space}@<T> && @\exposconcept{compound-scales-with}@<T, T>;

template<typename T>
concept @\deflibconcept{field_number_line}@ = @\libconcept{field_number}@<T> && @\libconcept{number_line}@<T>;

template<typename T>
concept @\deflibconcept{scalar_number}@ = @\libconcept{field_number}@<T> && (@\libconcept{field_number_line}@<T> || is_complex_number_v<T>);
\end{itemdecl}

\begin{itemdescr}
\pnum
\begin{note}
The \libconcept{field_number} concept is modeled by types
that behave similarly to \tcode{std::complex<double>}.
It represents an approximation of a field, see \refiev{102-02-18}, Note 2 to entry.
\end{note}

\pnum
\begin{note}
The \libconcept{field_number_line} concept is modeled by types
that behave similarly to \tcode{double}.
\end{note}

\pnum
\begin{note}
\libconcept{scalar_number} represents an approximation of a scalar number\irefiev{102-02-18}.
\end{note}
\end{itemdescr}
