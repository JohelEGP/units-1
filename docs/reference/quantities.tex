%!TEX root = std.tex
\Sec0[qties]{Quantities library}

\Sec1[qties.summary]{Summary}

\pnum
This Clause describes components for dealing with quantities,
as summarized in \tref{qties.summary}.

\begin{modularlibsumtab}{Quantities library summary}{qties.summary}
\ref{qty.concepts}      & Concepts                               & \tcode{<mp-units/quantity.h>} \\
\end{modularlibsumtab}

\Sec1[qty.syn]{Header \tcode{<mp-units/quantity.h>} synopsis}

\indexheader{mp-units/quantity.h}%
\indexlibraryspec{number_scalar}{quantity}%
\begin{codeblock}
#include <mp-units/numbers.h>  // see \ref{nums.syn}

namespace mp_units {

// \ref{qty.concepts}, concepts
template<class>
concept scalar_quantity = @\seebelow@;

template<Quantity Q>
struct number_scalar<Q> : number_scalar<typename Q::rep> {};

}  // namespace mp_units
\end{codeblock}

\Sec1[qty.concepts]{Concepts}

\begin{itemdecl}
template<class T>
concept @\deflibconcept{scalar_quantity}@ = Quantity<T> && @\libconcept{scalar_number}@<typename T::rep>;
\end{itemdecl}

\begin{itemdescr}
\pnum
\begin{note}
The \libconcept{scalar_quantity} concept
represents an approximation of a scalar quantity\irefiev{102-02-19}.
\end{note}
\end{itemdescr}
